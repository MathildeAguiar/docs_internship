

\textbf{Etudiant (nom et prénom) :} AGUIAR Mathilde \\
\textbf{Année d’étude dans la spécialité:} INFO4 \\

\textbf{Entreprise :} LORIA \\
\textbf{Adresse complète : (géographique et postale) :} Loria \\
Campus Scientifique \\
BP 239 \\
54506 Vandoeuvre-lès-Nancy \\
\textbf{Téléphone (standard) :} 03 83 59 20 00 \\

\textbf{Responsable administratif (nom et fonction) :} Cecilia Auvachez, Gestion des Ressources Humaines \\
\textbf{Téléphone:} +33 (0)3 83 59 30 65 \\
\textbf{Courriel:} cecilia.auvachez@loria.fr \\

\textbf{Tuteur de stage (organisme d’accueil) :} PARMENTIER Yannick\\
\textbf{Téléphone:} +33 3 54 95 86 58 \\
\textbf{Courriel:} yannick.parmentier@loria.fr \\

\textbf{Enseignant-référent:} MONIN Jean-François \\
\textbf{Téléphone:} 04 57 42 22 31 \\
\textbf{Courriel:} jean-francois.monin@univ-grenoble-alpes.fr \\

\textbf{Titre : (maximum 2 à 3 lignes).} \\
Analyse et développement d’une application web dans le cadre d’un projet
d’apprentissage des langues assisté par ordinateur (Computer-Aided Language
Learning) \\
\textbf{Résumé : (minimum 15 lignes).}\\
Pour mon stage d’INFO4 j’ai eu la chance d’intégrer le LORIA et l’équipe de recherche SyNaLP. J’ai travaillé avec M.Parmentier, mon tuteur de stage, pour l’assister dans le développement d’une application d’aide à l’apprentissage des langues. Cette application a pour but de faciliter la création d’exercices ciblant des notions grammaticales particulières en proposant un analyseur de textes qui va auto-générer des questions (traitant ladite notion). 
Mon rôle était de développer la partie "enseignant" de l’application. Pour ce faire j’ai utilisé le framework léger Flask ainsi que certains plugins comme Flask WTForms, Bootstrap Flask , Flask SQLAlchemy et ses dialectes SQLite. J’ai à la fois créé les interfaces Web mais je me suis aussi occupée de redéfinir et recoder la base de données. 
Dans une seconde partie je me suis occupée de la partie traitement des langues. Pour cela j'ai dû créer un jeu de données à l'aide de textes et articles directement extraits via des techniques de Web scraping de WikiData et de la librairie libre Gutenberg. J'ai ensuite dû réaliser un modèle capable de réaliser des tâches de topic labelling pour déterminer les sujets principaux contenus dans des textes passés en entrée. 


